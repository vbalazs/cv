%%%%%%%%%%%%%%%%%%%%%%%%%%%%%%%%%%%%%%%%%
% Medium Length Professional CV
% LaTeX Template
%
% This template has been downloaded from:
% http://www.LaTeXTemplates.com
%
% Original author:
% Trey Hunner (http://www.treyhunner.com/)
%
% Important note:
% This template requires the resume.cls file to be in the same directory as the
% .tex file. The resume.cls file provides the resume style used for structuring the
% document.
%
%%%%%%%%%%%%%%%%%%%%%%%%%%%%%%%%%%%%%%%%%

%----------------------------------------------------------------------------------------
%	PACKAGES AND OTHER DOCUMENT CONFIGURATIONS
%----------------------------------------------------------------------------------------

\documentclass{resume} % Use the custom resume.cls style

\usepackage{enumitem}
\usepackage[utf8]{inputenc}
\usepackage[colorlinks=true,urlcolor=blue]{hyperref}
\usepackage[left=0.75in,top=0.6in,right=0.75in,bottom=0.6in]{geometry} % Document margins
\setdescription{leftmargin=2cm,labelindent=1cm}

\name{Varga Balázs} % Your name
\address{Budapest \\ Dummy Str. 10.} % Your address
\address{+3600~$\cdot$~000~$\cdot$~0000 \\
\href{mailto:nospam@nospam.com}{nospam@nospam.com}} % Your phone number and email

\begin{document}

%----------------------------------------------------------------------------------------
%	EDUCATION SECTION
%----------------------------------------------------------------------------------------

\begin{rSection}{Tanulmányok}

{\bf Budapesti Műszaki és Gazdaságtudományi Egyetem, Budapest} \hfill {\em 2008 szeptembertől} \\
Folyamatban lévő mérnök informatikus BSc képzés \smallskip \\
Simonyi Károly Szakkollégium
\end{rSection}

%----------------------------------------------------------------------------------------
%	WORK EXPERIENCE SECTION
%----------------------------------------------------------------------------------------

\begin{rSection}{Tapasztalatok}

\begin{rSubsection}{INBUSS Kft.}
{2011 júliustól}{Szoftverfejlesztő}{Budapest}
\item Feladat: szoftverek fejlesztése, tervezése, testreszabása, fejlesztői támogatás
\item Projektek:
	\begin{description}
		\item[Negoma webshop] \hfill \\
		Funkciók implementálása, hibajavítás, szkriptek írása, új funkciók tervezése \\
		Használt technológiák: SVN, Tapestry 4, JavaEE, Hibernate, Perl, Javascript, JQuery, XML, Maven, PostgreSQL
		\item[Sun OpenSSO] \hfill \\
		Szoftvertámogatás, modulok fejlesztése, hibajavítása, felület testreszabása
		\item[DIDb - Driver Identification Database] \hfill \\
		Új funkciók fejlesztése, tervezése, tesztelése \\
		Elvégeztem a komplex projekthierarchia teljes migrálását Antról Mavenre \\
		Használt technológiák: SVN, JavaEE, Hibernate, Swing, Maven, Glassfish Metro, PostgreSQL \\
	\end{description}
\end{rSubsection}

%------------------------------------------------

\begin{rSubsection}{Simonyi Károly Szakkollégium}
{2009 márciustól}{Szoftverfejlesztő, üzemeltető}{Budapest}
\item Feladat: Projektmenedzsment, JavaEE alkalmazások fejlesztése és üzemeltetése
\item 2009 PHP alkalmazások fejlesztése
\item 2010-től Java, JavaEE ismeretek
\item 2010-2012: KIR Fejlesztők és Üzemeltetők csoport vezetése
\item 2012-től projektmenedzsment és fejlesztés
\item 2013 február: részvétel segítőként az mloc.js nemzetközi large scale Javascript konferencián
\item Projektek:
	\begin{description}
		\item[VIR Profil és Körök] \hfill \\
		Villanykari címtár, közösségi tevékenységek, értékelések nyilvántartása \\
		Projektvezetés, új funkciók fejlesztése, alkalmazás üzemeltetése \\
		Használt technológiák: Git, JavaEE, Hibernate, Wicket, SOAP, PostgreSQL
		\item[OpenAM és OpenDJ] \hfill \\
		Egységes, egyszeri bejelentkezés nyújtása a Villanykari webes szolgáltatások számára \\
		Alkalmazások konfigurációja, üzemeltetése, egyedi modulok fejlesztése
		\item[MasatMon] \hfill \\
		A Masat-I által küldött csomagokban lévő információk feldolgozása és megjelenítése \\
		Projektvezetés, tervezés, fejlesztés alatt \\
		Használt technológiák: Git, TrueCrypt, REST, SOAP, Glassfish, JavaEE, Math4j, Maven, MySQL
		\item[Simonyi Címtár] \hfill \\
		Szakkollégiumi tagságok tárolása, kereshetősge, nyilvántartása, szakkollégium specifikus funkciók \\
		Projektvezetés, tervezés, fejlesztés, üzemeltetés \\
		Használt technológiák: Git, Ruby, Rails, JQuery, Shibboleth, Apache2, Passenger, PostgreSQL
		\item[Printer.sch] \hfill \\
		Kollégiumban elérhetővé tett nyomtatók listázása \\
		Alkalmazás tervezése, fejlesztése, üzemeltetése \\
		Használt technológiák: Git, PHP, Shibboleth, MySQL
	\end{description}
\end{rSubsection}

%------------------------------------------------

\begin{rSubsection}{AB Connection Kft.}{2009 január - 2010 január}{PHP és JavaScript fejlesztő}{Budapest}
\item Feladat: egy OO PHP-ban írt saját backoffice rendszer továbbfejlesztése, különös
tekintettel a könnyű használhatóságra.
\item A használt technológiák, keretrendszerek: PHP5, MySQL, Subversion, JQuery, ExtJS
\end{rSubsection}

%------------------------------------------------

\end{rSection}

\begin{rSection}{Motiváció}
 Szeretnék jobban belemélyedni a Unit tesztek írásába és szeretnék nálam sokkal tapasztaltabbakkal dolgozni, hogy bővítsem az ismereteim.
\end{rSection}

%----------------------------------------------------------------------------------------
%	TECHNICAL STRENGTHS SECTION
%----------------------------------------------------------------------------------------

\begin{rSection}{Egyéb}

\begin{tabular}{ @{} >{\bfseries}l @{\hspace{6ex}} l }
Nyelvtudás & Angol (írás, olvasás: haladó) \\
Programozási nyelvek & Java (+++), JS (++), Ruby (+), C/C++ (+) \\
Protokollok \& APIs & TCP/IP alapok, XML, JSON, SOAP, REST \\
Adatbázisok & MySQL, PostgreSQL \\
Keretrendszerek & JavaEE, Wicket, JQuery \\
Eszközök & SVN, Git, Vim, NetBeans, Linux, Glassfish, Maven \\
Internetes profilok, blog & \href{http://vbalazs.me}{http://vbalazs.me}
\end{tabular}

\end{rSection}

%----------------------------------------------------------------------------------------
%	EXAMPLE SECTION
%----------------------------------------------------------------------------------------

%\begin{rSection}{Section Name}

%Section content\ldots

%\end{rSection}

%----------------------------------------------------------------------------------------

\end{document}
